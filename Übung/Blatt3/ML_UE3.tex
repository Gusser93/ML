\documentclass[a4paper,11pt,twoside]{article}
\usepackage[T1]{fontenc}
\usepackage{subcaption}
\usepackage[utf8]{inputenc}
\usepackage{ngerman, eucal, mathrsfs, amsfonts, bbm, amsmath, amssymb, stmaryrd,graphicx, array, geometry, color, wrapfig}
\geometry{left=25mm, right=15mm, bottom=25mm}
\setlength{\parindent}{0em} 
\setlength{\headheight}{0em} 
\title{Machine Learning\\ Blatt 3}
\author{Markus Vieth, David Klopp, Christian Stricker}
\date{\today}
\input{../head/lstlisting.tex}
\begin{document}

\newcommand{\cor}[1]{\textcolor{red}{\textit{#1}}}
\maketitle
\cleardoublepage
\pagestyle{myheadings}
\markboth{Markus Vieth,  David Klopp, Christian Stricker}{Markus Vieth, David Klopp, Christian Stricker}

\section*{Nr.1}
Bagging erwartet einen Parameter, der angibt wie viele Bäume erzeugt werden sollen. Jeder dieser Bäume betrachtet hierbei alle Attribute. \\
Bei Random Forest handelt es sich um eine Unterform des Bagging, die bei der Erstellung der Bäume nicht alle Attribute, sondern nur eine Auswahl derer berücksichtigt. Diese m vielen Attribute werden vor der Erstellung eines Baumes zufällig aus der Menge aller Attribute ausgewählt. \\Wenn n die Anzahl aller Attribute ist, so entspricht m bei der Klassifizierung in der Regel $\sqrt n$.\\
Random Forest erzeugt somit höchstens Teilbäume der Tiefe m, wohingegen beim Bagging die Tiefe bis zu m betragen kann. 
Dies führt beim Random Forest zu einer erhöhten Geschwindigkeit bei der Erstellung eines Baumes.
\section*{Nr.2}

\section*{Nr.3}

\end{document}