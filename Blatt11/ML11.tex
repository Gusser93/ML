\documentclass[a4paper,11pt,twoside]{scrartcl}
\usepackage[T1]{fontenc}
\usepackage{subcaption}
\usepackage[utf8]{inputenc}
\usepackage{ngerman, eucal, mathrsfs, amsfonts, bbm, amsmath, amssymb, stmaryrd,graphicx, array, geometry, color, wrapfig, float, hyperref}
\geometry{left=25mm, right=15mm, bottom=25mm}
\setlength{\parindent}{0em} 
\setlength{\headheight}{0em} 
\title{Machine Learning\\ Blatt 11}
\author{Markus Vieth\and David Klopp\and Christian Stricker}
\date{\today}
\input{../head/lstlisting.tex}
\begin{document}

\newcommand{\cor}[1]{\textcolor{red}{\textit{#1}}}
\maketitle
\cleardoublepage
\pagestyle{myheadings}
\markboth{Markus Vieth,  David Klopp, Christian Stricker}{Markus Vieth, David Klopp, Christian Stricker}

\newpage

\section*{Nr.1}
\subsection*{Skalarprodukt}
\subsubsection*{Symmetrie}
\[ k(x,y) = <x,y> = \sum_{i=1}^{n} x_i\cdot y_i = \sum_{i=1}^{n} y_i\cdot x_i = <y,x> = k(y,x) \]
\subsubsection*{zz. $\forall c_1, \ldots, c_m \in \mathbb{R}: \sum_{i,j = 1}^m c_ic_jk(x_i, x_j) \ge 0$}
\[ \sum_{i,j = 1}^{m} c_i c_j k(x_i, x_j) = \sum_{i,j = 1}^{m} \left( c_ic_j\sum_{k=1}^{n} ({x_i}_k, {x_j}_k) \right) = \sum_{k = 1}^{n}\sum_{i=1}^{m}\sum_{j=1}^{m}(c_i{x_i}_k \cdot c_j{x_j}_k)  \]
\[ = \sum_{k=1}^{n}\left( \sum_{i=1}^{m} c_i{x_i}_k \cdot \sum_{j=1}^{m}c_j{x_j}_k \right) = \sum_{k=1}^{n}\left( \sum_{i=1}^{m}c_i{x_i}_k \right)^2 \ge 0 \]
$\Rightarrow ~ <x,y>$ ist ein Kernel.
\begin{flushright}
	q.e.d.
	\end{flushright}
\subsection*{Polynomieller Kernel}
\[ k(x, y) = (\sum_{i=1}^n x_i y_i)^2 = \sum_{i=1}^n x_i y_i \cdot \sum_{i=1}^n x_i y_i = <x,y> \cdot <x,y> \]
$\Rightarrow$ Kernel, da Produkt 2er Kernel (wie oben gezeigt).
\begin{flushright}
	q.e.d.
\end{flushright}
\end{document}