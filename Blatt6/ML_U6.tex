\documentclass[a4paper,11pt,twoside]{scrartcl}
\usepackage[T1]{fontenc}
\usepackage{subcaption}
\usepackage[utf8]{inputenc}
\usepackage{ngerman, eucal, mathrsfs, amsfonts, bbm, amsmath, amssymb, stmaryrd,graphicx, array, geometry, color, wrapfig, float, hyperref}
\geometry{left=25mm, right=15mm, bottom=25mm}
\setlength{\parindent}{0em} 
\setlength{\headheight}{0em} 
\title{Machine Learning\\ Blatt 6}
\author{Markus Vieth\and David Klopp\and Christian Stricker}
\date{\today}
\input{../head/lstlisting.tex}
\usepackage{pifont}
\renewcommand\thefootnote{\ding{\numexpr171+\value{footnote}}}
\begin{document}

\newcommand{\cor}[1]{\textcolor{red}{\textit{#1}}}
\maketitle
\cleardoublepage
\pagestyle{myheadings}
\markboth{Markus Vieth,  David Klopp, Christian Stricker}{Markus Vieth, David Klopp, Christian Stricker}

\newpage

\section*{Nr.1}
\begin{align*}
 MSE( g_T(x_0) ) &= Var(g_T(x_0)) + Bias(f(x_0)) \\
 &= \mathbb{E_T}\left[ \left( g_T(x_0) - \mathbb{E_T}( g_T(x_0) ) \right)^2 \right] + \left( \mathbb{E_T}(g_T(x_0)) - f(x_0) \right)^2\\
 &= \mathbb{E_T}\left[ \left( g_T(x_0) - \mathbb{E_T}( g_T(x_0) ) \right)^2 \right] + 2\underset{=\mathbb{E_T}(g_T(x_0)) - \mathbb{E_T}(g_T(x_0)) = 0}{\underbrace{\mathbb{E_T}\left( g_T(x_0) - \mathbb{E_T}(g_T(x_0)) \right)}}\left( \mathbb{E_T}(g_T(x_0)) - f(x_0) \right) \\
 &+ \left( \mathbb{E_T}(g_T(x_0)) - f(x_0) \right)^2\\
 &= \mathbb{E_T}\left[ \left( g_T(x_0) - \mathbb{E_T}( g_T(x_0) ) \right)^2 \right] + 2\mathbb{E_T}\left[\left( g_T(x_0) - \mathbb{E_T}(g_T(x_0)) \right)\overset{\footnotemark}{\overbrace{\left( \mathbb{E_T}(g_T(x_0)) - f(x_0) \right)}}\right] \\
 &+ \mathbb{E_T}\left[\overset{\footnotemark}{\overbrace{\left( \mathbb{E_T}(g_T(x_0)) - f(x_0) \right)^2}}\right]\\ 
  &= \mathbb{E_T}\left[ \left( g_T(x_0) - \mathbb{E_T}( g_T(x_0) ) \right)^2  + 2\left(\left( g_T(x_0) - \mathbb{E_T}(g_T(x_0)) \right)\left( \mathbb{E_T}(g_T(x_0)) - f(x_0) \right)\right)\right . \\
  &\left .+ \left( \mathbb{E_T}(g_T(x_0)) - f(x_0) \right)^2\right]\\
  &= \mathbb{E_T}\left[ \left(\left( g_T(x_0) - \mathbb{E_T}( g_T(x_0) ) \right) + \mathbb{E_T}(g_T(x_0)) - f(x_0)\right)^2 \right]\\
  &= \mathbb{E_T}\left[ \left( g_T(x_0) - f(x_0) \right)^2 \right]
\end{align*}
\begin{flushright}
	q.e.d.
\end{flushright}
\addtocounter{footnote}{-1}
\footnotetext{Konstant, kann also in den Erwartungswert gesetzt werden}
\stepcounter{footnote}
\footnotetext{Konstant, somit ist der Erwartungswert er selbst}


\section*{Nr.2}
$X$ = \begin{tabular}{ c  c  c  c }
$3.437$ & $5.791$ & $3.268$ & $10.649$ \\
$12.801$ & $4.558$ & $5.751$ & $14.375$ \\
$6.136$ & $6.223$ & $15.175$ & $2.811$ \\
$11.685$ & $3.212$ & $0.639$ & $0.964$ \\
$5.733$ & $3.22$ & $0.534$ & $2.052$ \\
$3.021$ & $4.348$ & $0.839$ & $2.356$ \\
$1.689$ & $0.634$ & $0.318$ & $2.209$ \\
$2.339$ & $1.895$ & $0.61$ & $0.605$ \\
$1.025$ & $0.834$ & $0.734$ & $2.825$ \\
$2.936$ & $1.419$ & $0.331$ & $0.231$ \\
$5.049$ & $4.195$ & $1.589$ & $1.957$ \\
$1.693$ & $3.602$ & $0.837$ & $1.582$ \\
$1.187$ & $2.679$ & $0.459$ & $18.837$ \\
$9.73$ & $3.951$ & $3.78$ & $0.524$ \\
$14.325$ & $4.3$ & $10.781$ & $36.863$ \\
$7.737$ & $9.043$ & $1.394$ & $1.524$ \\
$7.538$ & $4.538$ & $2.565$ & $5.109$ \\
$10.211$ & $4.994$ & $3.081$ & $3.681$ \\
$8.697$ & $3.005$ & $1.378$ & $3.338$ \\
\end{tabular}
%
$Y$ = \begin{tabular}{ c  c  c  c }
$27.698$ \\
$57.634$ \\
$47.172$ \\
$49.295$ \\
$24.115$ \\
$33.612$ \\
$9.512$ \\
$14.755$ \\
$10.57$ \\
$15.394$ \\
$27.843$ \\
$17.717$ \\
$20.253$ \\
$37.465$ \\
$101.334$ \\
$47.427$ \\
$35.944$ \\
$45.945$ \\
$46.89$ \\
\end{tabular}
\newline
\newline
\newline
Berechne:
\[w = (X^T \cdot X)^{-1} \cdot (X^T \cdot Y)\]
mit:\\
$(X^T \cdot X)^{-1}$ = \begin{tabular}{ c  c  c  c }
$0.00429129055565$ & $-0.00478345506973$ & $-0.000752238199424$ & $-0.000689936030722$ \\
$-0.00478345506973$ & $0.0109403979337$ & $-0.00209111752592$ & $0.000396499981558$ \\
$-0.000752238199424$ & $-0.00209111752592$ & $0.00558456810802$ & $-0.00081147325675$ \\
$-0.000689936030722$ & $0.000396499981558$ & $-0.00081147325675$ & $0.000930021643118$ \\
\end{tabular}\\
\newline
\newline
$(X^T \cdot Y)$ = \begin{tabular}{ c }
$5570.426016$ \\
$2944.414095$ \\
$2902.209741$ \\
$6296.28324$ \\
\end{tabular}
\newline
\newline
\newline
Es ergibt sich somit:
$w$ = \begin{tabular}{ c }
$3.29265832646$ \\
$1.99479384949$ \\
$0.750919342272$ \\
$0.824836613555$ \\
\end{tabular}
\newline
\newline

$Out(X) =$ Catlle $\cdot 3.29265832646~+$ Calves $\cdot 1.99479384949~+$ Pigs $\cdot 0.750919342272~+$ Lambs $\cdot 0.824836613555$

\end{document}