\documentclass[a4paper,11pt,twoside]{scrartcl}
\usepackage[T1]{fontenc}
\usepackage{subcaption}
\usepackage[utf8]{inputenc}
\usepackage{ngerman, eucal, mathrsfs, amsfonts, bbm, amsmath, amssymb, stmaryrd,graphicx, array, geometry, color, wrapfig, float, hyperref}
\geometry{left=25mm, right=15mm, bottom=25mm}
\setlength{\parindent}{0em} 
\setlength{\headheight}{0em} 
\title{Machine Learning\\ Blatt 10}
\author{Markus Vieth\and David Klopp\and Christian Stricker}
\date{\today}
\input{../head/lstlisting.tex}
\begin{document}

\newcommand{\cor}[1]{\textcolor{red}{\textit{#1}}}
\maketitle
\cleardoublepage
\pagestyle{myheadings}
\markboth{Markus Vieth,  David Klopp, Christian Stricker}{Markus Vieth, David Klopp, Christian Stricker}

\newpage

\section*{Nr.1}
Sei $x_1$ eine Instanz der Klasse 1 und $x_2$ eine Instanz der Klasse 2.\\
Hyperebene: $f(x) = (w\cdot x) + b$\\
Bedingung: $|y((w\cdot x) +b)| = 1$\\
Größe der Margin: $\frac{w}{||w||} \cdot (x_1 - x_2)$\\
Herleitung des Optimierungsproblems:
\[ f(x_1) - f(x_2) = 2 = w\cdot x_1 + b - w\cdot x_2 - b = w\cdot(x_1-x_2) \]
\[ \Rightarrow \text{zu maximieren: }\frac{w}{||w||}\cdot(x_1 - x_2) = \frac{2}{||w||} \]
Daraus folgt, dass je eine Instanz pro Klasse ausreicht, um das Optimierungsproblem der maximum-margin Hyperplane zu definieren und somit eine eindeutige Lösung zu erhalten. 
\end{document}